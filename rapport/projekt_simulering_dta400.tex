\documentclass[a4paper,11pt,notitlepage,twocolumn]{article}
\usepackage[T1]{fontenc}
\usepackage[utf8]{inputenc}
\usepackage{lmodern}
\usepackage{graphicx}

\title{Performance of a gateway node in an emotional recognition system for saftey alerts.}
\author{Alexander Norberg}
% \newcommand\cTITLE{Performance of a gateway node in an emotional recognition system for saftey alerts.}
% \newcommand\cAUTHOR{Alexander Norberg}
% \newcommand\cEMAIL{alexander.norberg@student.hv.se}
% \newcommand\cDEPARTMENT{Department of Engineering Science}
% \newcommand\cSCHOOL{University West}

\begin{document}

%   \begin{titlepage}
%     \begin{flushleft}
%       \includegraphics[scale=0.3]{universitywest_svenska.eps}
%     \end{flushleft}
%     \begin{center}
%       \vspace*{5cm}
%       \Huge
%       \textbf{\cTITLE}\\
%       \vspace{3cm}
%       \Large
%       \textbf{\cAUTHOR}\\
%       \large
%       \textbf{\cEMAIL}
%       \vfill
%       \vspace{0.8cm}
%     \end{center}
%     \begin{flushright}
%       \cDEPARTMENT\\
%       \cSCHOOL\\
%       \today\\
%     \end{flushright}
%   \end{titlepage}
%
  \maketitle
  \begin{abstract}
    I have together with my group partner Artin Nadjati simulated the performance of a
    gateway node in a fog network which is supposed to handle emotional recognition on
    images sent from edge nodes which contain an observed object or place and be able to
    deduct if a crime has happened or if someone is in need of medical attention. The
    gateway node also sends the information together with GPS-coordinates of the edge
    node to a cloud server for history keeping and statistical analysis.
  \end{abstract}
  
  \section{Introduction}
    With the rise in machine learning and techniques to be able to recognize faces in
    software the idea of automatically detect when the law enforcement, emergency medical
    service or the fire department is needed and send an alert is getting more possible.
    We have decided to simulate a system similar to the system used by Bhattacherjee, Kumar
    and Rajalakshmi in[1] but have it as an automated system instead of needing a button
    press.
  \section{Methods}
    To make this simulation we use simpy[2] which is a library written in python for use in
    in python to simulate discrete events. We also decided to use the system resources used
    in the article[1] which was a Raspberry PI 3 B+ model connected to an Arduino Pro Mini
    with a camera sensor attached for taking pictures. The Raspberry is also connected to a
    cloud server used to store the results data together with GPS coordinates of that data.
    Note that the images themselves are not stored only the results of the image and
    emotional recognition, for privacy reasons. For the time data we used this[3] article as
    source, they also used a RaspberryPI to first do some machine training and then did a test
    to see the performance of the recognition which was about 1.21 fps or about 825 ms per
    picture to do a facial recognition. As for the simulation itself it is built with a 
    simpy resource with one resource per thread, which is four on the Raspberry PI. The
    cloud is also built as a single resource which the gateway calls every ten data points
    which it received and calculated. The incoming data from the edge nodes are sent at the
    interval time of ten seconds divided by number of devices, as the devices send their data
    of five pictures every ten seconds which is a picture taking every 2 seconds on the edge
    device. The simulation did not account for the times it took the LoRa network as used in[1]
    to send the data. The calculation time in the simulation is a simpy even sleep which sleeps
    for an amount of discrete steps. At the end of the simulation the total time it takes for
    a batch of pictures to be sent, waiting for it to be processed and the time to process
    the data is plotted on a graph. The source code can be viewed online on[4] or read at
    as an appendix at the end of this article.
  
  \section{Results}
    The results show that the sweet spot in this configuration seems to be nine edge nodes
    connected to a single gateway node. With ten edge nodes the gateway is not able to keep
    up with the incoming data as can be seen in Figure 1. And with eigth edge nodes the gateway is
    running without any waiting time which shows in Figure 2. With nine nodes the gatway node is
    showing some times when it is not able to keep up but gets it down to a lower level also
    instead of the waiting times getting higher all the time which is contained in Figure 3. 
    
  \section{Discussion}
    When at ten devices the processing time will be so long that the data which the gate receives
    will just build up and the time it takes for each batch of pictures will we longer each time.
    With this simulation basing the processing power of a Raspberry PI 3 there exists at the time
    of writing this article the successor Raspberry PI for with more computing power which
    might be one solution to be able to handle more edge nodes. The other might be to have lower
    resolution camera but that might impact how good the recognition software can identify
    the threat it is looking for.
    
  \section{Conclusion}
    Through this simulation we found out that a gateway node consisting of a Raspberry PI can
    handle about nine edge devices in this constellation. With 
    
  \section{References}
    

  \onecolumn
  \section{Appendix - source code.}
\end{document}
