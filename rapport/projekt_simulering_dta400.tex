\documentclass[a4paper,11pt,notitlepage]{article}
\usepackage[T1]{fontenc}
\usepackage[utf8]{inputenc}
\usepackage{lmodern}
\usepackage{graphicx}

\title{Performance of a gateway node in an emotional recognition system for saftey alerts.}
\author{Alexander Norberg}
% \newcommand\cTITLE{Performance of a gateway node in an emotional recognition system for saftey alerts.}
% \newcommand\cAUTHOR{Alexander Norberg}
% \newcommand\cEMAIL{alexander.norberg@student.hv.se}
% \newcommand\cDEPARTMENT{Department of Engineering Science}
% \newcommand\cSCHOOL{University West}

\begin{document}

%   \begin{titlepage}
%     \begin{flushleft}
%       \includegraphics[scale=0.3]{universitywest_svenska.eps}
%     \end{flushleft}
%     \begin{center}
%       \vspace*{5cm}
%       \Huge
%       \textbf{\cTITLE}\\
%       \vspace{3cm}
%       \Large
%       \textbf{\cAUTHOR}\\
%       \large
%       \textbf{\cEMAIL}
%       \vfill
%       \vspace{0.8cm}
%     \end{center}
%     \begin{flushright}
%       \cDEPARTMENT\\
%       \cSCHOOL\\
%       \today\\
%     \end{flushright}
%   \end{titlepage}
%
  \maketitle
  \twocolumn
  \begin{abstract}
    I have together with my group partner Artin Nadjati simulated the performance of a
    gateway node in a fog network which is supposed to handle emotional recognition on
    images sent from edge nodes which contain an observed object or place and be able to
    deduct if a crime has happened or if someone is in need of medical attention. The
    gateway node also sends the information together with GPS-coordinates of the edge
    node to a cloud server for history keeping and statistical analysis.
  \end{abstract}
  
  \section{Introduction}
    With the rise in machine learning and techniques to be able to recognize faces in
    software the idea of automatically detect when the law enforcement, emergency medical
    service or the fire department is needed and send an alert is getting more possible.
    We have decided to simulate a system similar to the system used by Bhattacherjee, Kumar
    and Rajalakshmi in[1] but have it as an automated system instead of needing a button
    press.
  \section{Methods}
    To make this simulation we use simpy[2] which is a library written in python for use in
    in python to simulate discrete events. We also decided to use the system resources used
    in the article[1] which was a Raspberry PI 3 B+ model connected to an Arduino Pro Mini
    with a camera sensor attached for taking pictures. The Raspberry is also connected to a
    cloud server used to store the results data together with GPS coordinates of that data.
    Note that the images themselves are not stored only the results of the image and
    emotional recognition, for privacy reasons.
  
  \section{Results}
    The results show that the sweet spot in this configuration seems to be nine edge nodes
    connected to a single gateway node. With ten edge nodes the gateway is not able to keep
    up with the incoming data. And with eigth edge nodes the gateway is running without any
    waiting time. With nine nodes the gatway node is showing some times when it is not able
    to keep up but gets it down to a reasonable level fast.
    
  \section{Discussion}
    Subjective interpretation of the results, describe what you think the show.
    "What does it mean?"
    
  \section{Conclusion}
    In the Conclusion section, state the most important outcome of your work. Do not simply
    summarize the points already made in the body — instead, interpret your findings at a
    higher level of abstraction. Show whether, or to what extent, you have succeeded in
    addressing the need stated in the Introduction.
    
  \section{References}
    

  \onecolumn
  \section{Appendix - source code.}
\end{document}
